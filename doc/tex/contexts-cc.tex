\subsection{Certificate Chain Context (cc)}
A certificate chain context is used to verify the trustchain of a user certificate by checking each certificate signature and asserting that the chain is attested by a trusted certificate authority.
\subsubsection{States} ~\\
\begin{contextstates}{Certificate Chain Context States}
clean* & Initial clean state. \\
invalid & Error state. \\
stale & CC context is stale. \\
linked & CC is linked. \\
checked & CC has been checked and verified. \\
\end{contextstates}
\subsubsection{Transitions}
\begin{contexttransitions}{Certificate Chain Context Transitions}
create &clean& linked & Create new certificate chain. \\
\tabucline[0.4pt on 0.4pt off 2pt]{-}
add\_certificate &linked& linked & Add new certificate to the certificate chain. \\
\tabucline[0.4pt on 0.4pt off 2pt]{-}
check &linked& checked & Check that the current root of the CC is a trusted CA certificate. \\
\tabucline[0.4pt on 0.4pt off 2pt]{-}
get\_last\_cert &linked& linked & Return the last certificate which is the current root of the CC. \\
\tabucline[0.4pt on 0.4pt off 2pt]{-}
get\_certificate &checked& checked & Return user certificate. \\
\tabucline[0.4pt on 0.4pt off 2pt]{-}
get\_not\_before &linked& linked & Return start of validity period. \\
\tabucline[0.4pt on 0.4pt off 2pt]{-}
get\_not\_after &linked& linked & Return end of validity period. \\
\tabucline[0.4pt on 0.4pt off 2pt]{-}
invalidate &*& invalid & Invalidate certificate chain; it can only be reused by explicitly resetting the context. \\
\tabucline[0.4pt on 0.4pt off 2pt]{-}
reset &*& clean & Reset certificate chain to initial clean state. \\
\end{contexttransitions}