\subsection{Diffie-Hellman Context (dh)}
A Diffie-Hellman context represents a Diffie-Hellman exchange with the peer.
\subsubsection{States} ~\\
\begin{contextstates}{Diffie-Hellman Context States}
clean* & Initial clean state. \\
invalid & Error state. \\
stale & DH context is stale. \\
created & Waiting for remote pubvalue. \\
generated & Diffie-Hellman shared secret has been calculated and is ready to be used. \\
\end{contextstates}
\subsubsection{Transitions}
\begin{contexttransitions}{Diffie-Hellman Context Transitions}
get\_dha\_id &created& created & Return DHA reference. \\
\tabucline[0.4pt on 0.4pt off 2pt]{-}
get\_secvalue &created& created & Return local Diffie-Hellman secret value. \\
\tabucline[0.4pt on 0.4pt off 2pt]{-}
consume &generated& clean & Use Diffie-Hellman shared secret thus consuming it. \\
\tabucline[0.4pt on 0.4pt off 2pt]{-}
create &clean& created & Create new Diffie-Hellman context. \\
\tabucline[0.4pt on 0.4pt off 2pt]{-}
generate &created& generated & Generate the shared Diffie-Hellman secret. \\
\tabucline[0.4pt on 0.4pt off 2pt]{-}
invalidate &*& invalid & Invalidate Diffie-Hellman context; it can only be reused by explicitly resetting the context. \\
\tabucline[0.4pt on 0.4pt off 2pt]{-}
reset &*& clean & Reset Diffie-Hellman context to initial clean state. \\
\end{contexttransitions}